\documentclass{article}

\usepackage{times}
\usepackage{graphicx}
\usepackage{subfigure}
\usepackage{natbib}
\usepackage{algorithm}
\usepackage{algorithmic}
\usepackage{lipsum}
\usepackage{todonotes}
\usepackage[accepted]{icml2017}

\usepackage{macros}

\newcommand{\bert}{\operatorname{BERT}}
\newcommand{\distortion}{\operatorname{Distortion}}
\newcommand{\prune}{\operatorname{Prune}}
\newcommand{\diffprune}{\operatorname{DiffPrune}}



\newcommand{\btheta}{{\bm{\theta}}}
\newcommand{\bphi}{{\bm{\phi}}}
\newcommand{\bpsi}{{\bm{\psi}}}

\icmltitlerunning{CS 287 Final Project Template}

\begin{document}

\twocolumn[
\icmltitle{BERT Model Compression and Weight Sharing}
\begin{icmlauthorlist}
  \icmlauthor{Jiafeng Chen}{}
\icmlauthor{Yufeng Ling}{}
\icmlauthor{Francisco Rivera}{}
\end{icmlauthorlist}

\vskip 0.3in
]

\begin{abstract}
  \begin{itemize}
  \item This document describes the expected style, structure, and rough proportions for your final project write-up.
  \item While you are free to break from this structure, consider it a strong prior for our expectations of the final report.
\item Length is a hard constraint. You are only allowed max \textbf{8 pages} in this format. While you can include supplementary material, it will not be factored into the grading process. It is your responsibility to convey the main contributions of the work in the length given.
  \end{itemize}



\end{abstract}

\section{Introduction}
\label{sec:introduction}

Example Structure:
\begin{itemize}
\item What is the problem of interest and what (high-level) are the current best methods for solving it?
\item How do you plan to improve/understand/modify this or related methods?
\item Preview your research process, list the contributions you made, and summarize your experimental findings.
\end{itemize}


\section{Background}

% Example Structure:
% \begin{itemize}
% \item What information does a non-expert need to know about the problem domain?
% \item What data exists for this problem?
% \item What are the challenges/opportunities inherent to the data? (High dimensional, sparse, missing data, noise, structure, discrete/continuous, etc?)
% \end{itemize}

Deep Neurel Networks (DNNs) have become extremely large in size with growing complexities. A BERT base model has 110M parameters and takes 438 MB disk space to store. This makes it very hard to deploy those models on mobile devices, which in most cases have limited storage and RAM and lower computational resources. The challenge that we are facing is how to compress DNNs in size while keeping their predictive performance competitive with full model.

\section{Related Work}

% Example Structure:
% \begin{itemize}
% \item What 3-5 papers have been published in this space?
% \item How do these  differ from your approach?
% \item What data or methodologies do each of these works use?
% \item How do you plan to compare to these methods?
% \end{itemize}

We are interested in compressing the language representation model BERT \cite{devlin2018bert}. It is the state-of-art result in many natural language processing tasks such as natural language inference, paraphrasing and named entity recognition.

\cite{cheng2017survey} provides a comprehensive overview of the state-of-art methods to compress DNNs. Of the methods summarized by the paper, the ones that are of interest to our work are weight-pruning and quantization \cite{han2015deep}, and low-rank factorization \cite{denton2014exploiting}. Knowledge distillation, first proposed by \cite{bucilua2006model} as ``knowledge transfer'', was recently popularized by \cite{hinton2015distilling}. It differs from the previously mentioned methods since it compresses an existing model by replacing it with a simpler model that is trained on its outputs.

DeepTwist, a end-to-end training framework, allows us to incorporate compression procedures into training \cite{lee2018deeptwist}. Under this framework, the model parameters are occasionally distorted during training time instead of only at the end. This framework has the benefit of simplified training procedure and faster convergence and reduced requirement for hyperparameter tuning. Experiment on MNIST dataset with DeepTwist shows that LeNet-5 and LeNet-300-100 models are able to retain predictive accuracy under high compression rate.

\section{Model}
\label{sec:model}

We first set up some notation. Let $F_\btheta = G_\bpsi \circ \bert_\bphi$ be a
transformed model where
$\btheta = (\bphi, \bpsi)$ be parameters of the subparts of the model.
$\bert_\bphi$ is a BERT model where the pretrained weights are $\bphi_0 = 
(\bphi_0^i)_{i=1}^n$, and
$G_\bpsi$ is additional layers for fine-tuning BERT, parameterized by $\bpsi$.
We decompose the subparts of the model into layers: \begin{align*}
\bert_\bphi &= f_{n}^{\bphi_n} \circ \cdots \circ f_1^{\bphi_1} \\
G_\bpsi &= g_{m}^{\bpsi_m} \circ \cdots \circ g_1^{\bpsi_1}.
\end{align*}

\subsection{DeepTwist-based Weight Distortions}





Example Structure:

\begin{itemize}
\item What is the formal definition of your problem?
\item What is the precise mathematical model you are using to represent it? In almost all cases this will use the probabilistic language from class, e.g.
  \begin{equation}
  z \sim {\cal N}(0, \sigma^2)\label{eq:1}
\end{equation}
But it may also be a neural network, or a non-probabilistic loss,
\[ h_t \gets \mathrm{RNN}(x_{t}, h_{t-1} )\]

This is also a good place to reference a diagram such as Figure~\ref{fig:diagram}.

\item What are the parameters or latent variables of this model that you plan on estimating or inferring? Be explicit. How many are there? Which are you assuming are given? How do these relate to the original problem description?
\end{itemize}



\begin{figure}
  \centering
  \missingfigure[figheight=8cm]{}
  \caption{\label{fig:diagram} This is a good place to include a diagram showing how your model works. Examples include a graphical model or a neural network block diagram.}
\end{figure}

\section{Inference (or Training)}
\label{sec:training}


\subsection{Knowledge Distillation}
\label{subsec:kd}

In this paper, we limit our purview to text classification tasks. Training
neural-network-based systems for this task typically consists of a form of
gradient descent based on a differentiable loss function which takes in the
prediction for a given example and the correct label. That is, if we have
training data $(\bm{x}_i, y_i)$ and model $f$, we would aim to minimize,
\[ \ell(f(\bm{x_i}), y_i)\]
where usually the loss function is taken to be Cross-Entropy loss. When we
already have a model $f$ that performs well, we can leverage this training
paradigm through a modification called \emph{knowledge distillation}. Given the
large model that performs well, termed the \emph{teacher network}, we train a
smaller network $g$, the \emph{student network}, by minimizing,
\[ \ell(g(\bm{x_i}), f(\bm{x_i})).\]
Here, we can no longer use Cross-Entropy loss because $f(\bm{x}_i)$ is a
probability distribution rather than a specific label. However, we can take
$\ell$ to be either mean-square error (MSE) between the probabilities in
log-space, or Kullback-Leibler (KL) loss. We avoid MSE error on the
probabilities themselves because of disappearing gradients when the predicted
probabilities are very close to 0 or 1.

The benefits of training in this fashion have been empirically documented
% TODO: paper on KD doing well?
and have theoretical motivation. Our goal of any network is to learn the
conditional distribution 
\[ p(y_i = k \mid \bm{x}_i).\]
In traditional training, we do this by observing the draws $y_i$ from this
distribution. However, to the extent that the teacher network prediction is a
good approximation of the true conditional distribution, the student network can
learn faster because it is being pushed toward predicting the de-noised
expectation.

To present a simplified example: if a neural network were trying to understand a
coin flip, it would require a number of observations of heads and tails to
conclude the underlying probability is 50\% toward either direction. However, if
it is directly shown that the probability was 50\% each time, regardless of what
the coin came up, it will in less time understand that the probability is always
50\%.

An additional benefit from knowledge distillation arises from BERT's
pre-training. Because BERT is pre-trained on large amounts of unstructured data,
it contains a powerful language model which likely would be impossible to learn
from the specific task's data-set. To an extent, relevant features from this
language model are what make BERT so powerful once it's fine-tuned. Training the
student network on BERT's output thus allows some of this ``outside knowledge''
to percolate toward the student network.

To provide a concrete example of how this might work, suppose BERT captures from
its pre-training that ``bad'', ``horrible,'' and ``terrible'' are approximate
synonyms. In a sentiment task, these three words appear in three different
sentences, two of which are negative, and one of which is positive. A fine-tuned
BERT may provide a lower likelihood of being positive to the positive sentence
than it would otherwise even if 


% TODO: KD benefits

For our knowledge distillation, we make use of a fine-tuned BERT on a task as
our teacher network and train a simpler model to emulate the predictions use
\nameref{sec:model}


% TODO: talk about KD

\subsection{Deep Twist}

% TODO: talk about DeepTwist™


\begin{comment}
\begin{itemize}
\item How do you plan on training your parameters / inferring the
  states of your latent variables (MLE / MAP / Backprop / VI / EM / BP / ...)

\item What are the assumptions implicit in this technique? Is it an approximation or exact? If it is an approximation what bound does it optimize?

\item What is the explicit method / algorithm that you derive for learning these parameters?
\end{itemize}
\end{comment}



\begin{algorithm}
  \begin{algorithmic}
    \STATE{}
  \end{algorithmic}
  \caption{Your Pseudocode}
\end{algorithm}




\section{Methods}

\begin{itemize}
\item What are the exact details of the dataset that you used? (Number of data points / standard or non-standard / synthetic or real / exact form of the data)

\item What are the exact details of the features you computed?


\item How did you train or run inference? (Optimization method / hyperparameter settings / amount of time ran / what did you implement versus borrow / how were baselines computed).

\item What are the exact details of the metric used?
\end{itemize}



\section{Results}

\begin{itemize}
\item What were the results comparing previous work / baseline systems / your systems on the main task?
\item What were the secondary results comparing the variants of your system?
\item This section should be fact based and relatively dry. What happened, what was significant?
\end{itemize}

\begin{table*}
  \centering
  \missingfigure{}
  \caption{This is usually a table. Tables with numbers are generally easier to read than graphs, so prefer when possible.}
  \label{fig:mainres}
\end{table*}


% \begin{table}
%   \centering
%   \missingfigure[figheight=5cm]{}
%   \caption{Secondary table or figure in results section.}
%   \label{fig:mainres}
% \end{table}



\section{Discussion}



\begin{itemize}
\item What conclusions can you draw from the results section?
\item Is there further analysis you can do into the results of the system? Here is a good place to include visualizations, graphs, qualitative analysis of your results.

\item  What questions remain open? What did you think might work, but did not?
\end{itemize}



\begin{figure}
  \centering
  \missingfigure{}
  \missingfigure{}
  \missingfigure{}
  \caption{Visualizations of the internals of the system.}
\end{figure}

\section{Conclusion}

\begin{itemize}
\item What happened?
\item What next?
\end{itemize}



% \section*{Acknowledgements}

% \textbf{Do not} include acknowledgements in the initial version of
% the paper submitted for blind review.

% If a paper is accepted, the final camera-ready version can (and
% probably should) include acknowledgements. In this case, please
% place such acknowledgements in an unnumbered section at the
% end of the paper. Typically, this will include thanks to reviewers
% who gave useful comments, to colleagues who contributed to the ideas,
% and to funding agencies and corporate sponsors that provided financial
% support.


\bibliography{paper}
\bibliographystyle{icml2017}

\end{document}
